\documentclass[aima203_lecturenotes_ku.tex]{subfiles}

\setcounter{chapter}{3}
\begin{document}
\chapter*{\textit{Chapter 4 Cont..} \hspace{15mm}  Spline Functions}
\section{Introduction}
Earlier we discussed the methods of finding an $nth$ order polynomial passing through $(n+1)$ points. In certain cases, it was found that these polynomials give \textbf{erroneous} result. Furthermore, it was found that a low-order polynomial approximation in each sub-interval provides a better approximation to the function than fitting a single high-order polynomial to the entire interval.
Such piece-wise connecting polynomials are called \textbf{spline functions}. The points at which two connecting splines meet are called knots.\\
The \textbf{cubic} spline is the most popular in engineering applications. Before starting cubic splines, we discuss linear and quadratic splines since such a discussion will eventually justify the development of cubic splines.

\subsection{Linear Splines}
Suppose the given data points are $(x_i,y_i), \; i=0,1,\, ..., \, n$ and  $h_i = x_i-x_{i-1} , \; i=1,\, 2, \, ..., \, n$. \\
Let $s_i(x)$ be a straight line from $x_{i-1}$ to $x_i$. Then, the slope of $s_i(x)$ is \\
$\displaystyle m_i = \frac{y_i - y_{i-1}}{x_i - x_{i-1}}$ and $s_i(x)= y_{i-1} + m_i(x-x_{i-1})$. \\[2mm]
From the discussion above the $s_i(x)$ are the \textbf{linear splines}.

\subsubsection{Drawback}
The linear splines derived above are continuous in $[x_0, x_n]$, but their slpes are discontinuous, i.e their first derivatives are discontinuous.

\subsection{Quadratic Splines}
Let $s_i(x)$ be a quadratic approximation of the data points in the sub-interval $[x_{i-1} -x_i]$ satisfying the following conditions:
\begin{enumerate}
\item $s_i(x)$ and $s_i'(x)$ are continuous on $[x_0,x_n]$,
\item $s_i(x_i)=y_i, \hspace{1cm} i=0,\, 1,\, 2, \, ..., \, n$
\item $s_i'(x) = \frac{1}{h_i} [ (x_i -x)m_{i-1} + (x-x_{i-1})m_i ]$ as $s_i'(x)$ is linear, where $m_i = s_i'(x)$.
\end{enumerate}
\textbf{Integrating} $s_i'(x)$ with respect to $x$, we obtain
\begin{equation}
  s_i(x) = \frac{1}{h_i} \left [ -\frac{(x_i -x)^2}{2}\, m_{i-1} + \frac{(x-x_{i-1})^2}{2} \, m_i \right ] + c_i
\end{equation}
Putting $x=x_{i-1}$ we get,
\begin{equation}
  c_i = y_{i-1} + \frac{h_i}{2}m_{i-1}
\end{equation}
Imposing the continuity condition on the spline functions $s_i(x)$ we get,
\begin{equation}
  m_{i-1} +m_{i} = \frac{2}{h_i}(y_i -y_{i-1}), \hspace{2cm} i = 1,\, 2, \, ..., \, n
\end{equation}
Imposing the natural spline condition $s''_1(x_1)=0$ we obtain, $ m_0 = m_1$

\subsubsection{Drawbacks}
The second derivative of the quadratic splines derived above are discontinuous which is an obvious disadvantages. This drawback is removed in the cubic splines.

\section{Cubic Splines}
When computers were not available, the draftsman used a device to draw a smooth curve through a given set of points such that the slope and the curvature are also continuous along the curve, that is $f(x), \; f'(x), \; f''(x)$ are continuous on the curve. Such a device was called a \textbf{spline} and plotting of the curve was called \textbf{spline fitting}. \\[1mm]
Let $s_i(x)$ be a cubic approximation of the data points in the sub-interval $[x_{i-1} -x_i]$ satisfying the following conditions:
\begin{enumerate}
\item $s_i(x)$ is atmost a cubic for $i = 1, \, 2, \, ..., \, n$,
\item $s_i(x)$, $s_i'(x)$ and $s_i''(x)$ are continuous on $[x_0,x_n]$,
\item $s_i(x)=y_i, \hspace{1cm} i=0,\, 1,\, 2, \, ..., \, n$
\item $\mathbf{s_i''(x) = \frac{1}{h_i} [ (x_i -x)M_{i-1} + (x-x_{i-1})M_i]}$ as $s_i''(x)$ is linear, where $M_i = s_i''(x)$.
\end{enumerate}
Integrating the condition-4, twice with respect to $x$: we get,
\begin{equation}
  \label{cubic_equation}
  s_i(x) = \frac{1}{h_i} \left [ -\frac{(x_i -x)^3}{6}\, M_{i-1} + \frac{(x-x_{i-1})^3}{6} \, M_i \right ] + c_i(x_i-x)+d_i(x-x_{i-1})
\end{equation}
Using the condition: $s_i(x_{i-1})=y_{i-1}$ and $s_i(x_i)=y_i$
\begin{equation}
    c_i = \frac{1}{h_i}\left [ y_{i-1} -  \frac{h_i^2}{6}M_{i-1} \right ], \hspace{10mm} d_i = \frac{1}{h_i}\left [ y_i -  \frac{h_i^2}{6}M_i \right ]
\end{equation}
Imposing all these conditions we get,
\begin{mdframed}
\begin{equation}
  \label{cubic}
  \frac{h_i}{6}\, M_{i-1} + \frac{1}{3}(h_i + h_{i+1})\, M_{i} + \frac{h_{i+1}}{6}\, M_{i+1} = \frac{y_{i+1} - y_i}{h_{i+1}} - \frac{y_i - y_{i-1}}{h_i}, \hspace{8mm} (i=1,2,...,n-1)
\end{equation}
\end{mdframed}

For subintervals of equal lengths:
\begin{equation}
  M_{i-1} + 4 M_i + M_{i+1} = \frac{6}{h^2}\, (y_{i+1} -2y_i + y_{i-1}), \hspace{2cm} (i=1,2,...,n-1)
\end{equation}
Imposing \textbf{natural spline} condition $s_i''(x_0) = s_i''(x_n) = 0$ in ~\ref{cubic}, we get,
\begin{align*}
  2(h_1+h_2)M_1 + h_2M_2 &= 6 \left[ \frac{y_2 - y_1}{h_2} - \frac{y_1 -y_0}{h_1} - h_1 M_0 \right ]\\[1mm]
  h_2M_1 + 2(h_2 + h_3)M_2 + h_3M_3 &= 6 \left[ \frac{y_3 - y_2}{h_3} - \frac{y_2 -y_1}{h_2}\right ] \\[1mm]
  h_3M_2 + 2(h_3 + h_4)M_3 + h_4M_4 &= 6 \left[ \frac{y_4 - y_3}{h_4} - \frac{y_3 -y_2}{h_3}\right ] \\[1mm]
  ... \\[1mm]
  h_{n-1}M_{n-2} + 2(h_{n-1} + h_n)M_{n-1} &= 6 \left[ \frac{y_n - y_{n-1}}{h_n} - \frac{y_{n-1} -y_{n-2}}{h_{n-1}}\right ] - h_n M_n
\end{align*}
This system is called \textbf{tridiagonal system} and there an efficient and an accurate method for solving it.

\begin{example}
  Obtain the natural cubic spline approximation for the function defined by the data: $(0,1), \; (1,2), \; (2,33), (3,244)$. Hence find an estimate of $y(2.5)$. \\[1mm]
  \textbf{Solution} \\[1mm]
  For the equally x-spaced data we obtain:
  \begin{gather}
    M_0 + 4M_1 + M_2 =6(y_2 -2y_1 + y_0) \\
    M_1 + 4M_2 + M_3 =6(y_3 -2y_2 + y_1)
  \end{gather}
  Using $M_0 = 0 = M_3$, we get, $4M_1 + M_2 = 180, \hspace{5mm} M_1 + 4M_2 = 1080$. Then,
  \begin{align*}
    s_3(x) &= \frac{1}{h_3} \left [ -\frac{(x_3 -x)^3}{6}\, M_2 + \frac{(x-x_2)^3}{6} \, M_3 \right ] + \\
    \frac{1}{h_3}\left [ y_2 -  \frac{h_3^2}{6}M_2 \right ](x_3-x)+ \frac{1}{h_3}\left [ y_3 -  \frac{h_3^2}{6}M_3 \right ](x-x_2) \\
           &=-46x^3+414x^2-985x+725 \\
    s(2.5) = -46(2.5)^3 =414(2.5)^2 -982(2.5) + 715 &= 121.25
  \end{align*}
\end{example}

\section{Cubic B splines}
The cubic spline formulae derived in the preceding section are global in nature, which means that they do not permit any local changes in the given data. The B-splines are the basis functions of the splines that facilitates us the local changes. \\
A B-spline of order $n$, is denoted by $s_{n,i}(x)$ is a spline of degree $(n-1)$ with knots $k_{i-n}, k_{i-n+1},...,k_i$, which is zero everywhere except in the interval $[k_{i-n}, k_i]$.
\begin{definition}
  A cubic B-spline $s_{4,i}(x)$ is a spline of degree $3$ with knots $k_{i-4}, k_{i-3},k_{i-2},k_{i-1},k_i$, such that
  \begin{enumerate}
  \item on each interval, $s_{4,i}(x)$ is a polynomial of degree $3$ or less,
  \item $s_{4,i}(x), \; s'_{4,i}(x), \; s''_{4,i}(x)$ are continuous, and
   \item $s_{4,i}(x) > 0$ inside $[k_{i-4}, k_i]$.
  \end{enumerate}
\end{definition}
\end{document}

%%% Local Variables:
%%% mode: LaTeX
%%% TeX-master: t
%%% End:
