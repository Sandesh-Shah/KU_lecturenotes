\documentclass[a4paper,twoside,12pt]{article}
\linespread{1}                     % For line spacing
\usepackage{amssymb}                 % For AMS symbol
\usepackage{amsmath,amsthm,latexsym}                 % This is useful for the matrix $\begin{bmatrix} a & b\\ 0 & c\end{bmatrix}$
\usepackage{verbatim}                % For comments in paragraph
\usepackage{graphicx}
\usepackage{epstopdf}
\usepackage{epsfig}              % For imbeding graph
\usepackage{color}                  % Color for the text
\usepackage[normal]{caption2}
\usepackage[utf8]{inputenc}
\usepackage[english]{babel}
\usepackage{multicol}
\setlength{\columnsep}{0.8cm}
\usepackage{makeidx}
%\usepackage[normal]{subfigure}
\usepackage{url}
\usepackage{lineno}
\usepackage{hyperref}
\usepackage{mathtools}
\usepackage{plain}
%%%%%%%%%%%%%
\textheight23cm%25.0cm
\textwidth16.3cm
\oddsidemargin-0.1cm
\evensidemargin0.1cm
%\topmargin-1cm
\headsep.3cm
%\usepackage{pstricks}
\usepackage{fancyhdr}
\pagestyle{fancy}
\fancyhf{}% Clear header/footer
\fancyhead[OC]{The Extended Sandwich Theorem; S. Thakuri}%Author on Odd page, Centred
\fancyhead[EC]{The Extended Sandwich Theorem; S. Thakuri }% Title on Even page, Centred
\fancyfoot[C]{\thepage}%

\title{\bfseries The Extension of the Sandwich theorem for limits}

\theoremstyle{plain}
\newtheorem{theorem}{Theorem}[section]
\newtheorem{lemma}[theorem]{Lemma}
\newtheorem{corollary}[theorem]{Corollary}
\newtheorem{algorithm}[theorem]{Algorithm}

\theoremstyle{definition}
\newtheorem{definition}[theorem]{Definition}
\newtheorem{example}[theorem]{Example}
\newtheorem{remark}[theorem]{Remark}

%---------------------------------------------------------------------------------------------------------------------------------------------------------------------------------
\begin{document}
\linenumbers
\vspace{2mm}
{\Large
\begin{center}
\bf{\LARGE \bfseries The Extension of the Sandwich theorem for limits}
\end{center}}
\begin{center}
Sandesh Thakuri$^{1}$, Bishnu Hari Subedi$^{2}$
\end{center}

\begin{center}
{\footnotesize
  $^{1}$Department of Artificial Intelligence, SoE, Kathmandu University, Nepal \\[1mm]
 \(^{2}\)Central Department of Mathematics, IoST, Tribhuvan University, Kirtipur, Nepal \\[2mm]
 Correspondence to: Sandesh Thakuri, Email: sandesh.775509@cdmath.tu.edu.np \\[1mm]
\textit{ 2020 Mathematics Subject Classification. 26A03}
}
\end{center}

\vspace{5mm}
\noindent
\textbf{Abstract:} {The scope of the Sandwich theorem for limit in analysis is limited to two sided-limits only. This paper extends the scope of the Sandwich theorem by considering one-sided limit as well and gives a more general version of  the Sandwich theorem by making the criterion of the Sandwich theorem more loose.}


The Sandwich theorem is simple yet powerful tool in analysis to determine and to analyze the limit of a function at a given point. We can leverage the known limits to calculate the unknown limits. Suppose, we know the limits of \(g(x)\) and \(h(x)\) at \(x=c\) to be the same limit \(L\) and here \(f(x)\) happens to be sandwich between \(g(x)\) and \(h(x)\) in some neighborhood of \(c\). Then we can conclude the limit of \(f(x)\) at \(x=c\) as \(L\) by the Sandwich theorem which is as follows.

\begin{theorem}[The Sandwich Theorem] \cite{thomas}
Suppose $g(x) \leq f(x) \leq h(x)$ in some open interval containing $c$, expect possibly at $x=c$ itself. Suppose also that $$\lim_{x \to c} g(x) = \lim_{x \to c} h(x) =L  .$$ Then $\displaystyle \lim_{x \to c} f(x)=L$.
\end{theorem}

\textbf{Keywords:} Limit, One-Sided Limit, Sandwich Theorem, Function.


\bibliographystyle{aomalpha}
\begin{thebibliography}{9}
    \bibitem{thomas}
Thomas, G. B. (2017) \emph{Thomas' Calculus}. Pearson India Education Services Pvt. Ltd, India.
\end{thebibliography}
\end{document}
