\documentclass[a4paper,twoside,12pt]{article}
\linespread{1}                     % For line spacing
\usepackage{amssymb}                 % For AMS symbol
\usepackage{amsmath,amsthm,latexsym}                 % This is useful for the matrix $\begin{bmatrix} a & b\\ 0 & c\end{bmatrix}$
\usepackage{verbatim}                % For comments in paragraph
\usepackage{graphicx}
\usepackage{epstopdf}
\usepackage{epsfig}              % For imbeding graph
\usepackage{color}                  % Color for the text
\usepackage[normal]{caption2}
\usepackage[utf8]{inputenc}
\usepackage[english]{babel}
\usepackage{multicol}
\setlength{\columnsep}{0.8cm}
\usepackage{makeidx}
\usepackage[normal]{subfigure}
\usepackage{url}
\usepackage{lineno}
\usepackage{hyperref}
\usepackage{mathtools}
\usepackage{plain}
%%%%%%%%%%%%%
\textheight23cm%25.0cm
\textwidth16.3cm
\oddsidemargin-0.1cm
\evensidemargin0.1cm
%\topmargin-1cm
\headsep.3cm
%\usepackage{pstricks}

\title{\bfseries A Variant of the Sandwich theorem for two-sided limits}


%---------------------------------------------------------------------------------------------------------------------------------------------------------------------------------
\begin{document}
\begin{center}
{\large  \bfseries Response to the Reviewer Comments on our manuscript The Extension of the Sandwich theorem for two-sided limits}
\end{center}
\vspace{5mm}
\begin{enumerate}
\item \textbf{Response on the comment titled ``Assessment of Novelty'' } \\[2mm]
  The reviewer has pointed out that the result is correct and reviewer has asked for the clarification on our claim that it is a ``Thakuri's Extension''. Here is our clarification:
  \begin{itemize}
  \item
    This is clarified on our revised manuscript submitted. While our theorem follows from the existing variants, our theorem has the novelty enough to call it an extension, because no such relaxation theorem exists for the Sandwich theorem for two-sided limits. \textbf{This has been acknowledge by the five reviewers of Asian Research Journal of Mathematics as well}, whose review details I can send if required. The reviewer has argued that this is not fundamentally new, the same reason we are using the word ``extension'' which means not entirely new but the criterion for the is relaxed.
        \vspace{5mm}

 \item We have revised our claim that to evaluate \(\displaystyle \lim_{x \to 0} \frac{sinx}{x}\) requires our theorem. This can be evaluated without the use of any Sandwich theorem also. We meant to say this cannot evaluated with the bounds we gave.
    \end{itemize}

\vspace{2mm}
\item \textbf{Response on the comment titled ``Key Observations'' } \\[2mm]
  The reviewer has pointed out that even though many textbooks do not include the one-sided version of the Sandwich theorem, the one-sided version aligns with the two sided version. \\[3mm]
  \begin{itemize}
  \item In this revised manuscript we have made the distinction that while our uses one-sided limits ours' is not one-sided version, ours' is two sided version that has incorporated the one-sided limits to relax the criterion.
    \vspace{5mm}
    \item The existing versions that has one-sided limits aligns with the classical Sandwich theorem, because they are the analogous theorem for the one-sided limit. But ours' isn't for the one-sided limits.
  \end{itemize}

\item \textbf{Response on the comment titled ``Key Observations'' } \\[2mm]
  \begin{itemize}
  \item We tried to rephrase the claim to ``variant'' but still we claim that the phrase ``extension'' suits better we our revised manuscript
    \vspace{5mm}

  \item The example $\frac{sinx}{x}$ has acknowledge the better bounds and its solution form the perspective our theorem.

    \vspace{5mm}
  \item We have cited, made reference and given credit to earler versions that led to our theorem. And we have given justification enough to call it an extension.
    \vspace{7mm}
 \item \textcolor{blue}{With the reviewers' insights} we were able to pose a question whether there exists a problem that can be solved cannot be solved by the existing versions of the Sandwich theorem but can be solved by our extension theorem.

 \vspace{7mm}
\item We have added all the references necessary, all the references that led to our work. With the reviewers suggestions we got a direction to dive into earlier works and contributions in this field. We got a chance to refine our article. The excuse for including only one reference is that we came up with this result with the help of that book only. We have given justice to this shortcoming in our revised manuscript. Thank you for the valuable review and the recommendations.

 \vspace{7mm}
\item Still if it feels like we have not justified to call our theorem \textbf{an extension} or if it feels more suitable to call it a \textbf{variant}, then we will send a final article with that change made immediately.
\end{itemize}

 \vspace{9mm}

  \begin{center}
    \LARGE Thank you.
  \end{center}
\end{enumerate}

\end{document}
%%% Local Variables:
%%% mode: LaTeX
%%% TeX-master: t
%%% End:
