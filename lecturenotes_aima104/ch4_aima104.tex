\documentclass[aima104_lecturenotes_ku.tex]{subfiles}

\setcounter{chapter}{3}
\begin{document}
\chapter{Orthogonality and Least Squares}
\section{Inner Product}
Let $u =
\begin{bmatrix}
  u_1 \\ u_2 \\ . \\ .\\ .\\ u_n
\end{bmatrix} $ and $v =
\begin{bmatrix}
  v_1 \\ v_2 \\ . \\ .\\ .\\ v_n
\end{bmatrix} $ be any two vectors in $\mathbb{R}^n$. Then the number $u^T\, v$ is called the \textbf{inner product} of $u$ and $v$. This inner product is also commonly known as \textbf{dot product} and denoted by $\mathbf{u.v}$.

\subsection{Properties of Inner Product}
\begin{mdframed}
  \begin{enumerate}
  \item $u.v = v.u$
  \item $(u+v).w = u.w + v.w$
  \item $(\alpha u).v = \alpha (u.v) = u. (cv)$
  \item $u.u \geq 0$ and $u.u = 0 \Longleftrightarrow u=0$
  \end{enumerate}
\end{mdframed}

\subsection{The Length of a Vector}
The length of a vector $v$ is called the \textbf{norm} of $v$. \\ It is denoted by $\Vert v \Vert$ and defined by $\Vert v \Vert = \sqrt{v_1^2 + v_2^2 + ... + v_n^2} $ so that, $\Vert v \Vert ^2 = v.v$ There are several kinds of norms actually, this particular norm is called \textbf{Euclidean norm}.
For any scalar $\alpha$, $\Vert \alpha v \Vert = \vert \alpha \vert \Vert v \Vert$. A vector whose length is unity is called a \textbf{unit vector}. If we divide a nonzero vector $v$ by its length, we obtain a unit vector $u$. This process is called \textbf{normalizing} of the vector $v$.

\subsubsection{Distance between vectors}
For $u$ and $v$ in a vector space $V$, the distance between them is written as dist(u,v) and is defined as $dist(u,v) = \Vert u-v \Vert$.

\subsection{Exercise}
\begin{enumerate}
\item Compute the following for the given vectors: \\
  $u=
  \begin{bmatrix}
    -1 \\ 2
  \end{bmatrix}, \vspace{4mm}
   v=
   \begin{bmatrix}
     4 \\6
   \end{bmatrix},
   \vspace{5mm}
   w=
   \begin{bmatrix}
     3 \\ -1 \\ 5
   \end{bmatrix},\vspace{4mm}
   x=
   \begin{bmatrix}
     6 \\ -2 \\3
   \end{bmatrix}
   $
   \begin{multicols}{2}
     a). $\displaystyle u.u, \;\; v.u, \;\; \frac{v.u}{u.u}, \;\; \Vert v \Vert $
     \columnbreak

     b). $\displaystyle w.w, \;\; x.w, \;\; \frac{x.w}{w.w}, \;\; \Vert x \Vert$
   \end{multicols}
 \item Find the distance between $u$ and $v$, and $w$ and $x$.
 \item Use matrix product and transpose definition to verify, property-2 and 3 of the inner product.
 \item Explain why $u.u \geq 0$. When is $u.u =0$.?
\end{enumerate}

\section{Orthogonal Vectors}
The two vectors $u$ and $v$ are orthogonal vectors if their dot product is zero,i.e $u.v = 0$. Observe that the zero vector is orthogonal to every vector as $0^Tv=0$ for all $v$.
\begin{thm}[The Pythagorean Theorem]
  Two vectors $u$ and $v$ are orthogonal if and only if $\Vert u + v \Vert ^2 = \Vert u \Vert ^2 + \Vert v \Vert ^2$.
\end{thm}

\subsection{Orthogonal Complement}
\begin{itemize}
\item If a vector $z$ is orthogonal to every vectors in a subspace $W$ then, $z$ is said to be orthogonal to $W$.

\item The set of all vectors that are orthogonal to $W$ is called the \textbf{orthogonal complement} of $W$. It is denoted by $W^{\perp}$. $W^{\perp} = \{z \, : \forall v \in W \;  z.v=0 \} $
\end{itemize}

\begin{thm}
  \begin{enumerate}
  \item A vector $x$ is in $W^{\perp}$ if and only if $x$ is orthogonal to every vector in a set that is spans $W$.
  \item $W^{\perp}$ is also a subspace.
   \item Row space is orthogonal complement of the Null space for a matrix.
  \end{enumerate}
\end{thm}

\subsection{Exercise}
\begin{enumerate}
 \item Verify parallelogram law: $\Vert u+v \Vert ^2 + \Vert u-v \Vert ^2 = 2 \Vert u \Vert ^2 + 2 \Vert v \Vert ^2$.
 \item Suppose $y$ is orthogonal to $u$ and $v$. Show that $y$ is orthogonal to every $w$ in $Span\{u,v\}$.
 \item Let $W$ be a subspace of $\mathbb{R}^n$, then show that $W^{\perp}$ a subspace of $\mathbb{R}^n$.
\item Show that if $x$ is in both $W$ and $W^{\perp}$, then $x=0$.
\end{enumerate}

\subsection{Orthogonal Sets}
A set of vectors $\{ u_1, ...u_p\}$ in a vector space $V$ is said to be \textbf{orthogonal set} if each pair of distinct vectors from the set is orthogonal,i.e for all $u_i, u_j \in V$ we have $u_i.u_j = 0$ whenever $i \neq j$.
\setcounter{theorem}{2}
\begin{mdframed}
\begin{theorem}
  Any orthogonal set is a linearly independent set.
\end{theorem}
\begin{definition}
  An orthogonal basis for a vector space $V$ is a basis for $V$ that is an orthogonal set.
\end{definition}
\begin{theorem}
  \label{orthopro}
  Let $\{u_1, ..., u_p\}$ be an orthogonal basis for a subspace $W$ of $\mathbb{R}^n$. For each $y$ in $W$, the coordinates of $y$ with respect to the orthogonal basis : $y = c_1u_1 + \; ... \; c_n u_n$ are given by $\displaystyle c_j = \frac{y.u_j}{u_j.u_j}$.
\end{theorem}
\end{mdframed}

\subsection{Orthogonal Projection}
The coordinate $c_j$ of $y$ in Theorem~\ref{orthopro} is actually orthogonal projection of $y$ into the vector $u_j$. This can be generalized. For any given vector $u$. The orthogonal projection of a vector $y$ on $u$ is given by the formula $\displaystyle \hat{y} = \frac{y.u}{u.u}\, u$

Or, it can be derived as follows using the inner-product. $(y-\alpha u)$ and $u$ are orthogonal so, $(y-\alpha u).u=0$. This gives us $\displaystyle \alpha = \frac{y.u}{u.u}$. For two dimensional vectors, another orthogonal component $z$ can be easily obtained as by subtracting the projection from the vector $y$. $z = y - \hat{y}$.

\section{Orthonormal Sets, {\mdseries \small \textcolor{red}{Important}}}
\begin{remark}
  A vector having length unity is called unit vector. Given a vector $v$ if we divide it by its magnitude which its length we obtain a unit vector along $v$, i.e $\displaystyle \frac{v}{\Vert v \Vert}$.
\end{remark}

\begin{mdframed}
  A set $\{u_1, ..., u_p\}$ is an orthonormal set if it is an orthogonal set of unit vectors. And a basis of orthonormal set is a orthonormal basis. The simplest orthonormal basis is $\{e_1, \; ...\; e_n$ for $\mathbb{R}^n$.
\end{mdframed}


Matrices whose columns form an orthonormal set are important in applications and in computer algorithms for matrix computations.

\subsection{Unitary Matrix}
A real matrix $U$ is said to be unitary if $U^tU=I$. If $U$ is complex then, $\overline{U}\, ^t U= I$, where $\overline{U}\, ^t$ is denoted by ${U}^{\dagger}$.
\begin{theorem}
  An $m \times n$ matrix $U$ has orthonormal columns if and only if $U^tU=I$
\end{theorem}
\subsection{Exercise}
\begin{enumerate}
\item Determine whether the following sets of vectors are orthogonal or not.
  \begin{enumerate}
  \item $
    \begin{bmatrix}
      1 \\ -2 \\ 1
    \end{bmatrix}, \vspace{5mm}
    \begin{bmatrix}
      0 \\ 1 \\ 2
    \end{bmatrix},  \vspace{5mm}
    \begin{bmatrix}
      -5 \\ -2 \\ 1
    \end{bmatrix}$
  \end{enumerate}

\item Show that $\{u_1, u_2\}$ is an orthogonal basis of $\mathbb{R}^2$ and find the coordinates of $x$ in terms of this basis.
  \begin{enumerate}
  \item   $u_1 =
  \begin{bmatrix}
    2 \\ 3
  \end{bmatrix} ,  \vspace{5mm}
  u_2 =
  \begin{bmatrix}
    6 \\ 4
  \end{bmatrix},  \vspace{5mm}
  x =
  \begin{bmatrix}
    9 \\ -7
  \end{bmatrix}
  $

 \item $u_1 =
  \begin{bmatrix}
    3 \\ 1
  \end{bmatrix} ,  \vspace{5mm}
  u_2 =
  \begin{bmatrix}
    2 \\ -6
  \end{bmatrix},  \vspace{5mm}
  x =
  \begin{bmatrix}
    -6 \\ 3
  \end{bmatrix}
  $

  \item Compute the orthogonal projection of $
    \begin{bmatrix}
      1 \\ 7
    \end{bmatrix}
    $ onto to the line through $
    \begin{bmatrix}
      -4 \\ 2
    \end{bmatrix}
    $ and the origin.

  \item Compute the distance of $ y
    \begin{bmatrix}
      3 \\ 1
    \end{bmatrix}
    $ to the line through
    $u= \begin{bmatrix}
      8 \\ 6
    \end{bmatrix}
    $ and the origin.
  \end{enumerate}
\item Take a square unitary matrix of order 2 and a vector $x \in \mathbb{R}^2$ and verify that \\ $\Vert Ux \Vert = \Vert x \Vert$. What can you infer from this property of a unitary matrix?

 \item Show that the Hadamard quantum gate defined by $H =
   \begin{bmatrix}
     1/\sqrt{2} & 1/\sqrt{2} \\[1mm]
     1/\sqrt{2} & -1/\sqrt{2}
   \end{bmatrix}
   $ is a unitary matrix.
 \end{enumerate}

 \section{The Gram-Schmidt Process}
 The Gram-Schmidt Process is a simple algorithm for producing an orthogonal or orthonormal basis for any nonzero subspace of $\mathbb{R}^n$. It uses the concept of \textbf{projection}. First we describe this process when a basis consisting of two vectors $\{x_1,x_2\}$ are given. Now we construct an orthogonal basis $\{v_1, v_2\}$ using the two vectors: $x_1$ and $x_2$ as follows:
 \begin{enumerate}
 \item Let $v_1 = x_1$.
  \item Draw the orthogonal projection from $x_2$ to $v_1$, which is given by $\displaystyle \frac{x_2.v_1}{v_1.v_1} \, v_1$. Now, from the theory of orthogonal projection given above, the vector perpendicular to $v_1$ is given by $\displaystyle v_2 = x_2 - \frac{x_2.v_1}{v_1.v_1} \, v_1$.
 \end{enumerate}

 For a given basis of $p$ vectors $\{x_1,x_2, ..., x_p\}$, we can continue the process of drawing the orthogonal projections and subtraction to get orthogonal vectors as follows:
 %\setcounter{enumerate}{2}
 \begin{enumerate}
 \item To obtain $v_3$ draw the orthogonal projection from $x_3$ onto the Span$v_1,v_2$ and \\[1mm] subtract this from $x_3$: $\displaystyle v_3 = x_3 - \frac{x_3.v_1}{v_1.v_1} \, v_1 - \frac{x_3.v_2}{v_2.v_2} \, v_2$.

  \item $\displaystyle v_p = x_p - \frac{x_p.v_1}{v_1.v_1} \, v_1 - \frac{x_p.v_2}{v_2.v_2} \, v_2 - ... - \frac{x_p.v_{p-1}}{v_1.v_{p-1}} \, v_{p-1}$.
  \end{enumerate}

  \begin{remark}
    After constructing orthogonal basis orthonormal basis can be easily constructed from the orthogonal basis. \textit{How?}
  \end{remark}

  \section{Exercise}
  \begin{enumerate}
  \item Find an orthonormal basis from the given vectors:
    \begin{multicols}{2}
    a). $\begin{bmatrix}
      3 \\ 0 \\ 1
    \end{bmatrix}, \vspace{5mm}
    \begin{bmatrix}
      8 \\ 5 \\ 6
    \end{bmatrix}
    $
    \columnbreak

b).$\begin{bmatrix}
      0 \\ 4 \\ 2
    \end{bmatrix}, \vspace{5mm}
    \begin{bmatrix}
      5 \\ 6 \\ 7
    \end{bmatrix}
    $
  \end{multicols}

\item Find an orthogonal basis for the column space of the matrix:
  \begin{multicols}{2}
    a).  $\begin{bmatrix}
        3 & -5 & 1 \\
        1 & 1 & 1 \\
        -1 & 5 &-2 \\
        3 & -7 & 8
    \end{bmatrix}$
    \columnbreak

    b).
    $\begin{bmatrix}
      -1 & 6 & 6 \\
      3 & -8 & 3 \\
      1 & -2 & 6 \\
      1 & -4 & -3
    \end{bmatrix}$
  \end{multicols}

\item Find the $QR$ factorization of the matrices given in the previous question.
\end{enumerate}

\subsection{QR Factorization}
\begin{theorem}
  If $A$ is an $m \times n$ matrix with linearly independent columns, then $A$ can be factored as $A=QR$, where $Q$ is an $m \times n$ matrix whose columns form an \textbf{orthonormal basis for $Col(A)$} and $R$ is an $n \times n$ \textbf{upper triangular matrix with positive entries on its diagonal}.
\end{theorem}

\begin{proof}
  Given linearly independent columns of the matrix $A$, we can construct $Q$ from the Gram-Schmidt process or any process. Then $Q$ being real unitary matrix we have $Q^t\, Q = I$. Then $Q^t\, A = Q^t\,Q R \implies R = Q^t\, A$.
\end{proof}
\section{Extra Materials}
The angle between the two vectors in plane and space can be generalized by the inner product. For two vectors $u,v \in \mathbb{R}^n$ we have, $u.v = \Vert u \Vert\, \Vert v \Vert \, cos\theta$, where $\theta$ is the angle between the two vectors: $u,v$.

\end{document}


%%% Local Variables:
%%% mode: LaTeX
%%% TeX-master: t
%%% End:
