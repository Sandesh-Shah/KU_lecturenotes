\documentclass[math101_lecturenotes_ku.tex]{subfiles}

\setcounter{chapter}{6}
\begin{document}

\chapter{System of Linear Equations}
The branch \textit{linear algebra} is much younger than the branch $calculus$. It is approximately 200 years old. The equation of the form $a_1x_1+a_2x_2+...+a_nx_n=b$ is called linear equation.

\section{Systems of Linear Equations}
A completely general system of $m$ linear equations in $n$ unknowns is of the following form:
\begin{gather*}
    a_{11}x_1+a_{12}x_2+...+a_{1n}x_n=b_1 \\
    a_{21}x_1+a_{22}x_2+...+a_{2n}x_n=b_2 \\
    .\\
    .\\
    .\\
    a_{m1}x_1+a_{m2}x_2+...+a_{mn}x_n=b_m
\end{gather*}
The $a_{1j}$ is the coefficient of $x_j$ in the $ith$ equation. The data for this system of equations are all the numbers $a_{ij}$ and $b_i$. Now consider the four matrices. \\[3mm]
$\displaystyle
A= \begin{bmatrix}
    a_{11} & a_{12} & ... & a_{1n}\\
    a_{21} & a_{22} & ... & a_{2n}\\
    .\\
    .\\
    .\\
    a_{m1} & a_{m2} & ... & a_{mn}
\end{bmatrix}$,
$\displaystyle
x= \begin{bmatrix}
    x_1 \\
    x_2 \\
    .\\
    .\\
    .\\
    x_n
\end{bmatrix}$,
$\displaystyle
b=\begin{bmatrix}
    b_1 \\
    b_2 \\
    .\\
    .\\
    .\\
    b_n
\end{bmatrix}$, \hspace{10mm}
$\displaystyle
[A|b]= \left[
    \begin{matrix}
    a_{11} & a_{12} & ... & a_{1n} \\
    a_{21} & a_{22} & ... & a_{2n}\\
    .\\
    .\\
    .\\
    a_{m1} & a_{m2} & ... & a_{mn}
\end{matrix} \;\;\; \vline \;\;\; \begin{matrix}
    b_1 \\
    b_2 \\
    .\\
    .\\
    .\\
    b_n
\end{matrix} \right ]
$ \\[5mm]
In this context of a system of equations; $A$ is called the coefficient matrix, $x$ is called vector of unknowns, $b$ is called righthandside vector, $[A|b]$ is called the augmented matrix.
So, the \textbf{\large matrix notation} of the system of linear equations is \textbf{$\displaystyle Ax=b$}. \\
\textit{Homework: Solve the following using matrix notation}:
\begin{align*}
    35x_1+6x_2-7x_3 =& 15 \\
    -90x_1-15x_2+21x_3 =& 15 \\
     25x_1+3x_2-7x_3 =& 15
\end{align*}

\subsection{Elementary Row operations}
Basically we have following three elementary row operations.

\begin{table}[ht!]
    \centering
    \begin{tabular}{|c|c|c|c|}
    \hline
        S.N & Operation & Description & Notation  \\
        \hline
        1. & replacement & Add a multiple of one row to another. & $r_i \leftarrow r_i+ar_j \;\;\; (i \neq j)$ \\
        2. & scale & Multiply a row by a nonzero factor. & $r_i \leftarrow cr_i \;\;\;(c\neq 0)$ \\
        3. & swap & Interchange a pair of rows. & $r_i \leftrightarrow r_j$\\
        \hline
    \end{tabular}
\end{table}
\vspace{-5mm}
\section{Echelon Form}
\subsection{Reduced Row Echelon Form}
With the help of the elementary row operations, any matrix can be transformed into a standard form called reduced row echelon form.
\begin{mdframed}[style=myframe]
        A matrix is in \textbf{reduced row echelon form} if
        \begin{itemize}
            \item All zero rows have been moved to the bottom of matrix.
            \item Each nonzero row has $1$ as its leading nonzero entry, using left-to-right ordering. Each such leading $1$ one is called a \textbf{pivot}.
            \item In each column containing a pivot, there is no other nonzero elements.
            \item The pivot in any row is farther to the right than the pivots in rows above.
        \end{itemize}
\end{mdframed}
\begin{enumerate}
    \item Here are four examples of matrices in reduced row echelon form: \\[2mm]
$\displaystyle \begin{bmatrix}
    0 & 1 \\
    0 & 0
\end{bmatrix}$, \hspace{8mm}
$\displaystyle \begin{bmatrix}
    1 & 0 \\
    0 & 1
\end{bmatrix}$, \hspace{10mm}
$\displaystyle \begin{bmatrix}
    0 & 1 & 3 & 0 \\
    0 & 0 & 0 & 1 \\
     0 & 0 & 0 & 0
\end{bmatrix}$, \hspace{15mm}
$\displaystyle \begin{bmatrix}
    1 & 5 & 0 & -7 & 0 \\
    0 & 0 & 1 & 3 & 0 \\
     0 & 0 & 0 & 0 & 1
\end{bmatrix}$

\item Here are four matrices not in reduced row echelon form. \\[2mm]
$\displaystyle \begin{bmatrix}
    0 & 0 \\
    1 & 0
\end{bmatrix}$, \hspace{8mm}
$\displaystyle \begin{bmatrix}
    0 & 1 \\
    1 & 0
\end{bmatrix}$, \hspace{10mm}
$\displaystyle \begin{bmatrix}
    0 & 1 & 3 & 2 \\
    0 & 0 & 0 & 1 \\
     0 & 0 & 0 & 0
\end{bmatrix}$, \hspace{15mm}
$\displaystyle \begin{bmatrix}
    0 & 1 & 3 & 0 \\
    0 & 0 & 0 & 4 \\
     0 & 0 & 0 & 0
\end{bmatrix}$
\end{enumerate}

\subsection{Row echelon form}
\begin{mdframed}[style=myframe]
        A matrix is in \textbf{row echelon form} if
        \begin{itemize}
            \item All zero rows have been moved to the bottom of matrix.
            \item The leading nonzero element in any row is farther to the right than the pivots in rows above.
            \item In each column containing a leading nonzero element, the entries below that leading nonzero element are zero.
        \end{itemize}
\end{mdframed}

\begin{enumerate}
    \item Notice the \textbf{staircase pattern} of the pivot positions.
    \item A row echelon form is obtain with less work than is required for the reduced row echelon form.
    \item The reduced row echelon form of a matrix is \textit{unique}, whereas a matrix may have many forms of row echelon forms.
\end{enumerate}
\begin{definition}
    A \textbf{pivot position} in a matrix is a location where a leading $1$ (a pivot) appears in the reduced row echelon form of that matrix. In general, we do not know the pivot positions until we have found the reduced row echelon form of the matrix, or any row echelon form.
\end{definition}

\subsection{Algorithm for the Reduced Row Echelon Form}
\begin{enumerate}
    \item Interchange the rows if necessary to place all zero rows on the bottom.
    \item Identify the leftmost nonzero column. Say it is pivot column $j$. Interchange rows to bring a nonzero element to the top row and $jth$ column, which is the pivot position. Use the row replacement operation to create zeros in all positions in the pivot column below the pivot position.

    \item Repeat Steps 1 and 2 on the remaining submatrix until there are no nonzero rows left. (\textit{We have a row echelon form}.)

    \item Beginning with the rightmost pivot, working upward and to the left, use row replacement operations to create zeros in all positions in the pivot column above the pivot position. Scale the entry in the pivot row to create a leading 1.

    \item Repeat Step 4, ending with the unique reduced row echelon form of the given matrix.
\end{enumerate}

\subsection{Exercise}
\begin{enumerate}
    \item Solve: \hspace{2mm} $3x_1+6x_2+6x_3=21, \hspace{5mm} 2x_1+4x_2+5x_3=16, \hspace{5mm}
    2x_1+5x_2+4x_3=17$

    \item Find all solutions: \hspace{2mm} $x-y+z=4, \hspace{3mm} 2x+y-3z=5, \hspace{3mm} -y+7x-3z=22$

    \item Find the general solution of the system,
    $x_1+3x_2+9x_3=6, \hspace{5mm}
    2x_1+7x_2+3x_3=-5, \hspace{5mm}
    x_1+4x_2-x_3=-11$
\end{enumerate}
\subsection{Uniquely and Parametrically represented solutions.}
In section we discuss how to tell whether a system of linear equations has a unique solution or many solutions which are parametrically represented solutions or has no solution. The case of having no solution is the case of inconsistency of the system, based on its reduced echelon form.

\begin{mdframed}
\begin{definition}[Rank of a matrix]
The rank of a matrix is the number of nonzero rows in its reduced row echelon form or row echelon form. We use the notation Rank($A$) for this number. So, the rank of a matrix is equal to the number of its pivots. \\[3mm]

So, it also defined as the number of pivot positions in the matrix.
\end{definition}
\end{mdframed}

\begin{remark}
  It is not necessary to carry out the reduction to reduced row echelon form to determine the pivot positions in a matrix. Reduction to row echelon form is sufficient for this.
\end{remark}

\begin{remark}
  This rank of a matrix is also called its \textbf{row rank} or its \textbf{column rank}.
\end{remark}

Let \(A\) be \(m \times n\) matrix.
\begin{enumerate}
  \item The rank of the coefficient matrix \(A\) equal to \(n\) then, it has a unique solution.
  \item The rank of the coefficient matrix \(A\) equals to the rank of \([A:b]\), but is less than \(n\) then, it has more than one solutions.
\item The rank of \(A\) is less than the rank of \([A:b]\) then, the system of equation is inconsistent. This is the case where the rank of \(A\) is less than \(n\) but there is a pivot position in the last column of the augmented matrix, \([A:b]\).
\end{enumerate}
\textbf{More than one solution:}
$$ \left[ \begin{matrix}
    1 & 2 & 3 \\
    4 & 5 & 6 \\
    7 & 8 & 9 \\
    10 & 11 & 12
\end{matrix} \;\; \vline \;\;
\begin{matrix}
    20\\ 47\\ 74\\ 101
\end{matrix} \right] \thicksim  \left[ \begin{matrix}
    1 & 0 & -1 \\
    0 & 1 & 2 \\
    0 & 0 & 0 \\
    0 & 0 & 0
\end{matrix} \;\; \vline \;\;
\begin{matrix}
    -2\\ 11\\ 0\\ 0
\end{matrix} \right]  \;\;\; \Longrightarrow \hspace{4mm}\begin{bmatrix}
    x_1\\ x_2 \\ x_3
\end{bmatrix} = \begin{bmatrix}
    -2\\ 11 \\0
\end{bmatrix}+ s \begin{bmatrix}
    1\\ -2 \\1
  \end{bmatrix}$$ \vspace{3mm}
  \textbf{Inconsistent System}
$$ \left[ \begin{matrix}
    2 & -4  \\
    4 & -1 \\
    1 & -1  \\
\end{matrix} \;\; \vline \;\;
\begin{matrix}
    3\\ 2\\ 1
\end{matrix} \right] \thicksim  \left[ \begin{matrix}
    1 & 0  \\
    0 & 1  \\
    0 & 0
\end{matrix} \;\; \vline \;\;
\begin{matrix}
    0\\ 0\\ -1
\end{matrix} \right]$$

\subsection{Exercise}
\begin{enumerate}
\item Find the rank of coefficient and augmented matrix and check the consistency of the given system of equations:
  \begin{enumerate}
  \item[a).]   \( \displaystyle 2x-6y+8z = 2, \hspace{4mm}  -4x+13y+3z = 6, \hspace{4mm} -6x+20y+14z = -2\)
  \item[b).] \( 2x-2y+4z+6w=8, \hspace{3mm} -4x+5y-2z-7w=-10, \hspace{3mm} 2x+y+22z+21w=10, \hspace{3mm} -3x+5y-4z+11w=10\)
  \end{enumerate}



\item Obtain the row rank and parametrically represented solution of the following systems.
  \begin{enumerate}
  \item[a).] \(\displaystyle -4x+12y-7z=8, \hspace{15mm} x-3y+2z=-1\)
   \item[b).] \(\displaystyle 3x_1-x_2+3x_3=5, \hspace{5mm} x_1+2x_2-3x_4=-1, \hspace{5mm} 2x_1+5x_2+4x_3+2x_4 = 10\).
  \end{enumerate}
\end{enumerate}
\end{document}

%%% Local Variables:
%%% mode: latex
%%% TeX-master: t
%%% End:
