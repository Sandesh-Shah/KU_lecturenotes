\documentclass[aima104_lecturenotes_ku.tex]{subfiles}

\setcounter{chapter}{1}
\begin{document}
\chapter{Linear Transformation}
\section{Introduction}
The terms \textit{function}, \textit{mapping}, \textit{map}, and \textit{transformation} are synonymous. Great part of the linear algebra is dedicated to the study of linear transformation.

\begin{mdframed}
    \begin{definition}
      A mapping $T$ from a vector space $V$ to a vector space $W$ is \textbf{linear} if $$T(\alpha x +\beta y) = \alpha T(x) + \beta T(y)$$ for all vectors $x, y \in V$ and for all scalars $\alpha,\, \beta$. \\[1mm]
      Equivalently, $T$ \textbf{linear} if,
      \begin{enumerate}[i)]
      \item $T(x + y) = T(x) + T(y)$ for all vectors $x, y \in V$, and
        \item $T(\alpha x) = \alpha T(x)$ for any scalars $\alpha$.
        \end{enumerate}
    \end{definition}
\end{mdframed}
\begin{example}
  Show that if $A$ is a $m \times n$ matrix then $T$ defined by $T(x)=Ax$ is a linear transformation from $\mathbb{R} ^n \to \mathbb{R} ^m$.
\end{example}

\subsection{Exercise}
\begin{enumerate}

\item Is the  map $f: \mathbb{R}^3 \to \mathbb{R}^3$ defined by $f(x_1,x_2,x_3)=(x_1+(x_2,3x_1-x_2+x_3,5x_1-x_3)$ a linear map? Explain.
\item Show that shift map is not a linear transformation.

\item Is there a linear transformation that maps $(1,0)$ to $(5,3,4)$ and maps $(3,0)$ to $(1,3,2)$?

\item Show that the transformation $T$ defined by $T(x_1,x_2)=(4x_1 - 2x_2, 3|x_2|)$ is not linear.
\item $T$ is defined by $T(x)=Ax$, find a vector $x$ whose image under $T$ is $b$, determine whether $x$ is unique.
  $$ A = \begin{bmatrix}
    1 & 0 & -2 \\
    -2 & 1 & 6 \\
    3 & -2 & 5
  \end{bmatrix}, \hspace{2cm}
  b= \begin{bmatrix}
    -1 \\ 7 \\ -3
  \end{bmatrix}
  $$

\item How many rows and columns must a matrix $A$ have in order to define a mapping from $\mathbb{R} ^4$ to $\mathbb{R} ^5$ by the rule $T(x) = Ax$?
\end{enumerate}

\section{Matrix of a Linear Transformation}
\begin{theorem}
  Let $A$ be an $m \times n$ matrix. The mapping $x \mapsto Ax$ is linear from $\mathbb{R}^n$ to $\mathbb{R}^m$.\\
  Conversely, for a linear map $T: \mathbb{R}^n \to \mathbb{R}^m$ there exists an $m \times n$ matrix $A$ such that $T(\vec{x})=A\vec{x}$ for all $\vec{x} \in \mathbb{R}^n$. In fact, $A$ is the $m \times n$ matrix whose $jth$ column is the vector $T(e_j)$, where $\{e_j\, : \, j=1,2,...,n\}$ is the the basis of the domain $\mathbb{R} ^n$,i.e $A= \left [ T(e_1)\; ... \; T(e_n)\right ]$.
\end{theorem}

\subsection{Exercise}
Find the standard matrix of the linear transformation $T$.
\begin{enumerate}
\item For the dilation transformation $T(x)=3x$, $x \in \mathbb{R}^2$.

\item Show that the transformation that rotates each point in $\mathbb{R}^2$, about the origin, through an angle $\phi$ counterclockwise, is a linear transformation by find the standard matrix of the transformation.

\item $T(x_1,x_2) = (2x_2-3x_1, x_1-4x_2, 0, x_2) $
\item $T:\mathbb{R} ^2 \mapsto \mathbb{R} ^2$ first reflects points through the $x_1$-axis then reflects points through the line $x_2 = x_1$
\item $T:\mathbb{R} ^2 \mapsto \mathbb{R} ^2$ that rotates each point through an angle $\phi$, with counterclockwise rotation.

\item Describe the transformation of the following matrices geometrically:
  $$\begin{bmatrix}
    1 & 0\\
    0 & -1
  \end{bmatrix}, \hspace{5mm} \begin{bmatrix}
    0 & 1 \\
    1 & 0
  \end{bmatrix}, \hspace{5mm} \begin{bmatrix}
    2 & 0 \\
    0 & 2
  \end{bmatrix}, \hspace{5mm} \begin{bmatrix}
    1 & 0 \\
    0 & 0
  \end{bmatrix}, \hspace{5mm} \begin{bmatrix}
    1 & 0 \\
    1 & 1
  \end{bmatrix}, \hspace{5mm} \begin{bmatrix}
    1 & 3 \\
    0 & 1
  \end{bmatrix}$$
\end{enumerate}
\begin{example}
  Describe the linear mapping that has the matrix $\begin{bmatrix}
    0 & -1 \\ 1 & 0
  \end{bmatrix}$. \\[1mm]
  Let the given matrix is denoted by $A$. $A$ is a $2 \times 2$ So, the map is from $\mathbb{R}^2 \mapsto \mathbb{R}^2$. Then for every $\vec{x}=(x_1,x_2) \in \mathbb{R}^2$, Now $A\vec{x}= (-x_2,x_1)$ which is a \textbf{rotation} transformation that rotates every vector through the angle $90^0$ counterclockwise.
\end{example}

\subsection{Types of Transformation}
Geometrically, linear transformation are basically of the following types:
\begin{enumerate}
\item Reflection
\item Rotation
\item Contraction and Expansion
\item Shears: Horizontal and Vertical
\item Projections
\end{enumerate}

\section{Kernel and Image of Linear Transformation}
Let $T:V \mapsto W$ be a linear transformation.
\begin{itemize}
\item \textbf{Kernal}\\[1mm]
  The kernal of $T$ is the set of all vectors in $V$ that maps to zero vector in $W$. It is denoted by \textit{Ker(T)}. $Ker(T)=\{v \in V\, ; \, T(v)=0\}$.In another words, Kernal is the \textit{Null Space} of $T$.
\item \textbf{Range} \\[1mm]
  The range of $T$ is the set of all vectors in $W$ which are the images of the vectors in $V$. It is denoted by $R(t)$. $R(T)=\{w \in W \, : \, w = T(v) \text{for some}  v \in V\}$.
\end{itemize}

\subsection{Exercise}
\begin{enumerate}
\item Show that $Ker(T)$ is a subspace of $V$ and $R(T)$ is a subspace of $W$.
\item  Given the vector space $V$ of all real-valued functions defined on an interval $[a,b]$ such that their first derivative functios are continous on $[a,b]$. Let $W$ be the vector space of all continuous functions of $[a,b]$. Show that $D: V \mapsto V$ that maps $f \in V$ to $f' \in w$ is a linear transformation and find the kernal of $D$.
\end{enumerate}

\subsection{Facts:}
\begin{enumerate}
\item The linear transformation $T: V \to W$ is one-to-one if, $dim(Ker(T) = 0$, \\
  i.e $Ker(T) = {0}$.

\item The linear transformation $T: V \to W$ is onto if, $dim(R(T) = dim(W)$.
\end{enumerate}

\section{Properties of Linear Transformation}
Let $T: \mathbb{R}^n \to \mathbb{R}^m$ be a linear transformation.
\begin{enumerate}
\item Then $T(0)=0$.
\item $T$ is one-to-one if and only if the equation $T(x)=0$ has only the trivial solution. (\textit{Indirectly it has got to do with the nullity or nullspace of the matrix.})
\item $T$ is onto if and only if the columns of the corresponding standard matrix spans the $\mathbb{R}^m$. (\textit{So it has got to do with the column space of the matrix.})
\end{enumerate}
Give an example of one-to-one and not one-to-one linear transformation.


\end{document}

%%% Local Variables:
%%% mode: LaTeX
%%% TeX-master: t
%%% End:
