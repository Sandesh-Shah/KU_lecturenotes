\documentclass[12pt]{exam}
\usepackage{preamble_question}
\usepackage[normalem]{ulem}
\renewcommand\ULthickness{2pt}
\setlength\ULdepth{1.3ex}

\begin{document}

\begin{center}
  {\bfseries  {\large Kathmandu University}} \\
  Second In-semester Examination-2025 \\[-2mm]
    \textbf{\small Department of Artificial Intelligence}
\end{center}
\vspace{2mm}

\begin{minipage}{0.70\linewidth}
  \begin{flushleft}
    Level: B.Tech Artificial Intelligence \\
    Year: II \\
    Time: 15 minutes
  \end{flushleft}
\end{minipage} \hfill
\begin{minipage}{0.25\linewidth}
  \begin{flushleft}
    Course: AIMA 203 \\
    Semester: II \\
    F.M. : 10
  \end{flushleft}
\end{minipage}
\vspace{-8mm}
\begin{center}
  \rule{\textwidth}{1pt}
  Name: \hspace{5cm}  Roll No: \hspace{3cm} Marks-Scored:
  \vskip -3mm
\rule{\textwidth}{1pt}
\end{center}



\begin{center}
  SECTION''A'' \hspace{5mm} [10 Q \(\times\) 1 = 10 marks] (\textit{Do any ten}).
\end{center}
Fill in the blanks by writing the most appropriate word(s) or symbol(s).
\begin{questions}
  \question Write the condition for the natural quadratic spline. \rule{7cm}{0.15mm}.
\vspace{2mm}
  \question Write the condition for the natural cubic spline. \rule{7cm}{0.15mm}.
\vspace{2mm}
  \question Write a major drawback of linear splines. \rule{7cm}{0.15mm}.
\vspace{2mm}
  \question Why is cubic spline the most popular one? \rule{7cm}{0.15mm} \\[1mm]
  \rule{9cm}{0.15mm}.
\vspace{2mm}
  \question What are $m_i$'s and $M_i$'s in the quadratic and cubic splines? \rule{5cm}{0.15mm}.
\vspace{2mm}
  \question Write the formula for Romberg's Integration using Trapezoidal rule. \rule{4cm}{0.15mm}.
\vspace{2mm}
  \question From where the $u_i$'s are obtained in Gaussian quadrature. \rule{5cm}{0.15mm}.
\vspace{2mm}
  \question Write an second-degree IVP. \rule{9cm}{0.15mm}.
\vspace{2mm}
\question Write an second-degree BVP. \rule{9cm}{0.15mm}.
\vspace{2mm}
  \question Difference between forward Euler method and backward Euler method: \rule{2cm}{0.15mm} \\[1mm]
    \rule{9cm}{0.15mm}.

\end{questions}

\end{document}


%%% Local Variables:
%%% mode: LaTeX
%%% TeX-master: t
%%% End:
